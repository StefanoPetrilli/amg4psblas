\section{User Interface\label{sec:userinterface}}
\markboth{\textsc{MLD2P4 User's and Reference Guide}}
         {\textsc{\ref{sec:userinterface} User Interface}}

The basic user interface of MLD2P4 consists of eight methods. The six 
methods \verb|init|, \verb|set|, \verb|build|, 
\verb|hierarchy_build|, \verb|smoothers_build| and \verb|apply|
encapsulate all the functionalities for the setup and the application
of any multilevel and one-level preconditioner implemented in the
package.  
The method \verb|free| deallocates the preconditioner data structure, while
\verb|descr| prints a description of the preconditioner setup by the user.
For backward compatibility,  methods are also accessible as
stand-alone subroutines.

For each method, the same user interface is overloaded with
respect to the real/ complex case and the single/double precision;
arguments with appropriate data types must be passed to the method,
i.e.,
\begin{itemize}
\item the sparse matrix data structure, containing the matrix to be
  preconditioned, must be of type \verb|psb_|\emph{x}\verb|spmat_type|
	with \emph{x} = \verb|s| for real single precision, \emph{x} = \verb|d|
	for real double precision, \emph{x} = \verb|c| for complex single precision,
	\emph{x} = \verb|z| for complex double precision;
\item the preconditioner data structure must be of type
  \verb|mld_|\emph{x}\verb|prec_type|, with \emph{x} =    
  \verb|s|, \verb|d|, \verb|c|, \verb|z|, according to the sparse
  matrix data structure;
\item the arrays containing the vectors $v$ and $w$ involved in
  the preconditioner application $w=B^{-1}v$ must be of type   
  \verb|psb_|\emph{x}\verb|vect_type| with \emph{x} =    
  \verb|s|, \verb|d|, \verb|c|, \verb|z|, in a manner completely
  analogous to the sparse matrix type;
\item real parameters defining the preconditioner must be declared
  according to the precision of the sparse matrix and preconditioner
  data structures (see Section~\ref{sec:precset}).
\end{itemize}
A description of each method is given in the remainder of this section.

\clearpage

\subsection{Method init\label{sec:precinit}}

\begin{center}
\verb|call p%init(ptype,info)|
\end{center}

\noindent
This method allocates and initializes the preconditioner
\verb|p|, according to the preconditioner type chosen by the user.

{\vskip1.5\baselineskip\noindent\large\bfseries Arguments} \smallskip

\begin{tabular}{p{1.2cm}p{12cm}}
%\verb|p|  & \verb|type(mld_|\emph{x}\verb|prec_type), intent(inout)|.\\
%              & The preconditioner data structure. Note that \emph{x}
%                must be chosen according to the real/complex, single/double
%                precision version of MLD2P4 under use.\\
\verb|ptype|  & \verb|character(len=*), intent(in)|.\\
              & The type of preconditioner. Its values are specified
              in Table~\ref{tab:precinit}.\\
              & Note that the strings are case insensitive.\\
\verb|info|   & \verb|integer, intent(out)|.\\
              & Error code. If no error, 0 is returned. See Section~\ref{sec:errors} for details.\\
%\verb|nlev|   & \verb|integer, optional, intent(in)|.\\
                % & The number of levels of the multilevel
                % preconditioner. This optional argument is deprecated,
                % new codes should set the number of levels with \verb|mld_precset|.\\
                % If \verb|nlev| is not present and \verb|ptype|=\verb|'ML'|, \verb|'ml'|, 
                % then \verb|nlev|=2 is assumed. Otherwise, \verb|nlev| is ignored.\\
\end{tabular}

\vskip1.5\baselineskip
For compatibility with the previous versions of MLD2P4, this method can be also invoked
as follows:

\begin{center}
\verb|call mld_precinit(p,ptype,info)|
\end{center}


\clearpage

\subsection{Method set\label{sec:precset}}

\begin{center}
\verb|call p%set(what,val,info [,ilev, ilmax, pos, idx])|
\end{center}

\noindent
This method sets the parameters defining the preconditioner \verb|p|. More
precisely, the parameter identified by \verb|what| is assigned the value
contained in \verb|val|. 

{\vskip1.5\baselineskip\noindent\large\bfseries Arguments} \smallskip

\begin{tabular}{p{1.2cm}p{12cm}}
%\verb|p|  & \verb|type(mld_|\emph{x}\verb|prec_type), intent(inout)|.\\
%              & The preconditioner data structure. Note that \emph{x} must
%                be chosen according to the real/complex, single/double precision
%                 version of MLD2P4 under use.\\
\verb|what|   & \verb|character(len=*)|. \\
              & The parameter to be set. It can be specified through its name;
                the string is case-insensitive. See
                Tables~\ref{tab:p_cycle}-\ref{tab:p_smoother_1}.\\ 
\verb|val |   & \verb|integer| \emph{or} \verb|character(len=*)| \emph{or}
                \verb|real(psb_spk_)| \emph{or} \verb|real(psb_dpk_)|,
                \verb|intent(in)|.\\
              & The value of the parameter to be set. The list of allowed
                values and the corresponding data types is given in
                Tables~\ref{tab:p_cycle}-\ref{tab:p_smoother_1}.
                When the value is of type \verb|character(len=*)|,
                it is also treated as case insensitive.\\
\verb|info|   & \verb|integer, intent(out)|.\\
              & Error code. If no error, 0 is returned. See Section~\ref{sec:errors}
                for details.\\
\verb|ilev|   & \verb|integer, optional, intent(in)|.\\
              & For the multilevel preconditioner, the level at which the
                preconditioner parameter has to be set.
                The levels are numbered in increasing
                order starting from the finest one, i.e., level 1 is the finest level.
                If \verb|ilev| is not present, the parameter identified by \verb|what|
                is set at all the appropriate levels (see
                Tables~\ref{tab:p_cycle}-\ref{tab:p_smoother_1}).\\
\verb|ilmax|   & \verb|integer, optional, intent(in)|.\\
              & For the multilevel preconditioner, when both
                \verb|ilev| and \verb|ilmax| are present, the settings
                are applied at all levels \verb|ilev:ilmax|. When
                \verb|ilev| is present but \verb|ilmax| is not, then
                the default is \verb|ilmax=ilev|.
                The levels are numbered in increasing
                order starting from the finest one, i.e., level 1 is the finest level. \\
\verb|pos|   & \verb|charater(len=*), optional, intent(in)|.\\
              & Whether the other arguments apply only to the pre-smoother (\verb|'PRE'|)
                or to the post-smoother (\verb|'POST'|). If \verb|pos| is not present,
                the other arguments are applied to both smoothers.
                If the preconditioner is one-level or the parameter identified by \verb|what|
                does not concern the smoothers, \verb|pos| is ignored.\\
\verb|idx|   & \verb|integer, optional, intent(in)|.\\
              & An auxiliary input argument that can be passed to the
                underlying objects.
\end{tabular}

\vskip1.5\baselineskip
For compatibility with the previous versions of MLD2P4, this method can be also invoked
as follows:

\begin{center}
\verb|call mld_precset(p,what,val,info)|
\end{center}

\noindent
However, in this case the optional arguments \verb|ilev|,
\verb|ilmax|, \verb|pos| and \verb|idx|
cannot be used. \\

A variety of preconditioners can be obtained
by a suitable setting of the preconditioner parameters. These parameters
can be logically divided into four groups, i.e., parameters defining
\begin{enumerate}
	\item the type of multilevel cycle and how many cycles must be applied;
        \item the aggregation algorithm;
        \item the coarse-space correction at the coarsest level (for multilevel
                 preconditioners only);
	\item the smoother of the multilevel preconditioners, or the one-level
                  preconditioner.
	
\end{enumerate}
A list of the parameters that can be set, along with their allowed and
default values, is given in Tables~\ref{tab:p_cycle}-\ref{tab:p_smoother_1}.
For a description of the meaning of the parameters, please
refer also to Section~\ref{sec:background}. \\

\textbf{Remark 2.} A smoother is usually obtained by combining two objects:
a smoother (\verb|SMOOTHER_TYPE|) and a local solver (\verb|SUB_SOLVE|),
as specified in Tables~\ref{tab:p_smoother}-\ref{tab:p_smoother_1}.
For example, the block-Jacobi smoother using
ILU(0) on the blocks is obtained by combining the block-Jacobi smoother
object with the ILU(0) solver object. Similarly,
the hybrid Gauss-Seidel smoother (see Note in Table~\ref{tab:p_smoother})
is obtained by combining the block-Jacobi smoother object with a single sweep
of the Gauss-Seidel solver object, while the point-Jacobi smoother is the
result of combining the block-Jacobi smoother object with a single sweep
of the pointwise-Jacobi solver object. However, for simplicity, shortcuts are
provided to set point-Jacobi, hybrid (forward) Gauss-Seidel, and
hybrid backward Gauss-Seidel, i.e., the previous smoothers can be defined
by setting only \verb|SMOOTHER_TYPE| to appropriate values (see
Tables~\ref{tab:p_smoother}), i.e., without setting
\verb|SUB_SOLVE| too.

The smoother and solver objects are arranged in a
hierarchical manner. When specifying a smoother object, its parameters,
including the local solver, are set to their default values, and when a solver
object is specified, its defaults are also set, overriding in both
cases any previous settings even if explicitly specified. Therefore if
the user sets a smoother, and wishes to use a solver
different from  the default one, the call to set the solver must come
\emph{after} the call to set the smoother. 

Similar considerations apply to the point-Jacobi, Gauss-Seidel and block-Jacobi
coarsest-level solvers, and shortcuts are available
in this case too (see Table~\ref{tab:p_coarse}). \\

\textbf{Remark 3.} In general, a coarsest-level solver cannot be used with
both the replicated and distributed coarsest-matrix layout;
therefore, setting the solver after the layout may change the layout.
Similarly, setting the layout after the solver may change the solver.

More precisely, UMFPACK and SuperLU require the coarsest-level
matrix to be replicated, while SuperLU\_Dist requires it to be distributed.
In these cases, setting the coarsest-level solver implies that
the layout is redefined according to the solver, ovverriding any
previous settings. MUMPS,  point-Jacobi,
hybrid Gauss-Seidel and block-Jacobi can be applied to
replicated and distributed matrices, thus their choice
does not modify any previously specified layout.
It is worth noting that, when the matrix is replicated,
the point-Jacobi, hybrid Gauss-Seidel and block-Jacobi solvers
reduce to the corresponding local solver objects (see Remark~2).
For the point-Jacobi and Gauss-Seidel solvers, these objects
correspond to a \emph{single} point-Jacobi sweep and a \emph{single} 
Gauss-Seidel sweep, respectively, which are very poor solvers.

On the other hand, the distributed layout can be used with any solver
but UMFPACK and SuperLU; therefore, if any of these two solvers has already
been selected, the coarsest-level solver is changed to block-Jacobi,
with the previously chosen solver applied to the local blocks.
Likewise, the replicated layout can be used with any solver but SuperLu\_Dist;
therefore, if SuperLu\_Dist has been previously set, the coarsest-level
solver is changed to the default sequential solver.

\textbf{Remark 4.}  The argument \verb|idx| can be used to allow finer
control for those solvers; for instance, by specifying the keyword
\verb|MUMPS_IPAR_ENTRY| and an appropriate value for \verb|idx|, it is
possible to set any entry in the MUMPS integer control array.
See also Sec.~\ref{sec:adding}. 
%The \verb|what,val| pairs described here are those of the predefined
%moother/solver objects; newly developed solvers may define new pairs
%according to their needs. 


\bsideways
\begin{center}
%\begin{tabular}{|p{5cm}|l|p{2.4cm}|p{2.5cm}|p{5cm}|}
\begin{tabular}{|p{3.6cm}|l|p{2.4cm}|p{2.4cm}|p{7.2cm}|}
\hline
\verb|what|              & \textsc{data type}        &  \verb|val|      &  \textsc{default}  &
\textsc{comments} \\ \hline
%\multicolumn{5}{|c|}{\emph{type of the multilevel preconditioner}}\\ \hline
%\verb|mld_ml_cycle_| \par 
\verb|'ML_CYCLE'|     & \verb|character(len=*)|
                         & \texttt{'VCYCLE'} \par \texttt{'WCYCLE'}   \par \texttt{'KCYCLE'} \par
                             \texttt{'MULT'} \par \texttt{'ADD'}
                         & \texttt{'VCYCLE'}
                         &Multilevel cycle: V-cycle, W-cycle, K-cycle, hybrid Multiplicative Schwarz,
                           and Additive Schwarz. \par
                           Note that hybrid Multiplicative Schwarz is equivalent to V-cycle and
                           is included for compatibility with previous versions of MLD2P4. \\ \hline
%\verb|mld_outer_sweeps_| \par 
 \verb|'OUTER_SWEEPS'| & \texttt{integer} &
                           Any integer \par number $\ge 1$  & 1 &
                           Number of multilevel cycles. \\ \hline
%\verb|mld_smoother_type_| \par \verb|SMOOTHER_TYPE|  & \verb|character(len=*)|
%                         & \texttt{'JACOBI'} \ \ \ \texttt{'BJAC'} \ \ \ \texttt{'AS'}
%                         & \texttt{'AS'}
%                         & Basic predefined one-level preconditioner
%                         (i.e., smoother): Jacobi, block Jacobi, AS. \\ \hline
%\verb|mld_smoother_pos_| \par \verb|SMOOTHER_POS| & \verb|character(len=*)|
%                         & \texttt{'PRE'} \ \ \ \texttt{'POST'} \ \ \ \texttt{'TWOSIDE'}
%                         & \texttt{'TWOSIDE'}
%                         & ``Position'' of the smoother: pre-smoother, post-smoother, 
%                           pre- and post-smoother. \\
%\hline
\end{tabular}
\end{center}
\caption{Parameters defining the multilevel cycle and the number of cycles to
be applied.
\label{tab:p_cycle}}                       
\esideways
                   
\bsideways
\begin{center}
%\begin{tabular}{|p{5cm}|l|p{2.4cm}|p{2.5cm}|p{5cm}|}
\begin{tabular}{|p{3.9cm}|l|p{2.3cm}|p{2.9cm}|p{6.9cm}|}
\hline
\verb|what|              & \textsc{data type}        &  \verb|val|      &  \textsc{default}  &
\textsc{comments} \\ \hline
%\multicolumn{5}{|c|}{\emph{aggregation algorithm}} \\ \hline
%\verb|mld_min_coarse_size_|  \par 
\verb|'MIN_COARSE_SIZE'| & \verb|integer|
                         & Any number \par $> 0$
                         & $\lfloor 40 \sqrt[3]{n} \rfloor$, where $n$ is the dimension
                            of the matrix at the finest level
                         & Coarse size threshold. The aggregation stops
                            if  the global number of variables of the
                            computed coarsest matrix
                            is lower than or equal to this threshold
                           (see Note).
%                         or \par
%                         the coarsening ratio is lower than or equal to
%                         its value (see \verb|mld_min_aggr_ratio_|), or \par
%                         the maximum number of levels is reached
%                         (see \verb|mld_n_prec_levs_|). 
                           \\ \hline  
%\verb|mld_min_cr_ratio_|  \par 
\verb|'MIN_CR_RATIO'| & \verb|real|
                         & Any number \par $> 1$
                         & 1.5
                         & Minimum coarsening ratio. The aggregation stops
                            if the ratio between the matrix dimensions
                            at two consecutive levels is lower than or equal to this
                            threshold (see Note).\\ \hline   
%\verb|mld_max_levs_|  \par 
\verb|'MAX_LEVS'| & \verb|integer|
                         & Any integer \par number $> 1$
                         & 20
                         & Maximum number of levels. The aggregation stops
                           if the number of levels reaches this value (see Note). \\ \hline  
%\verb|mld_par_aggr_alg_|  \par 
\verb|'PAR_AGGR_ALG'|  & \verb|character(len=*)| \hspace*{-3mm}
                         & \texttt{'DEC'}, \texttt{'SYMDEC'}
                         & \texttt{'DEC'}
                         & Parallel aggregation algorithm. \par Currently, only the
                         decoupled aggregation (\verb|DEC|) is available; the
                           \verb|SYMDEC| option  applies decoupled
                           aggregation to  the sparsity pattern
                           of $A+A^T$.\\ \hline
%\verb|mld_aggr_type_|  \par 
\verb|'AGGR_TYPE'|  & \verb|character(len=*)| \hspace*{-3mm}
                         & \textbf{\texttt{'SOC1'}} &
                                                      \textbf{\texttt{'SOC1'}},
                                                      \textbf{\texttt{'SOC2'}}   
                         & Type of aggregation algorithm: currently,
                           we implement to measures of strength of
                           connection, the one by Van\v{e}k, Mandel
                           and Brezina~\cite{VANEK_MANDEL_BREZINA},
                           and the one by Gratton et al~\cite{GrHeJi:16}. \\ \hline
%\verb|mld_aggr_prol_|  \par 
\verb|'AGGR_PROL'|  & \verb|character(len=*)| \hspace*{-3mm}
                         & \texttt{'SMOOTHED'}, \texttt{'UNSMOOTHED'} & \texttt{'SMOOTHED'}
                         & Prolongator used by the aggregation algorithm: smoothed or unsmoothed
                         (i.e., tentative prolongator). \\
\hline
\multicolumn{5}{|l|}{{\bfseries Note.} The aggregation algorithm stops when
at least one of the following criteria is met: 
the coarse size threshold, the} \\
\multicolumn{5}{|l|}{minimum coarsening ratio, or the maximum number
of levels is reached. Therefore, the actual number of levels may be} \\
\multicolumn{5}{|l|}{smaller than the specified maximum number
of levels. } \\
\hline
\end{tabular}
\end{center}
\caption{Parameters defining the aggregation algorithm.
\label{tab:p_aggregation}} 
\esideways

\bsideways
\begin{center}
%\begin{tabular}{|p{5cm}|l|p{2.4cm}|p{2.5cm}|p{5cm}|}
\begin{tabular}{|p{3.8cm}|l|p{2.5cm}|p{2.3cm}|p{6.6cm}|}
\hline
\verb|what|              & \textsc{data type}        &  \verb|val|      &  \textsc{default}  &
\textsc{comments} \\ \hline
%\verb|mld_aggr_ord_|  \par 
\verb|'AGGR_ORD'|  & \verb|character(len=*)|
                         & \texttt{'NATURAL'} \par \texttt{'DEGREE'}
                         & \texttt{'NATURAL'}
                         & Initial ordering of indices for the aggregation
                            algorithm: either natural ordering or sorted by
                            descending degrees of the nodes in the
                            matrix graph. \\ \hline 
                            %Since aggregation is
                            %heuristic, results will be different.
%\verb|mld_aggr_thresh_| \par 
\verb|'AGGR_THRESH'| & \verb|real(|\emph{kind\_parameter}\verb|)|
                         & Any~real \par number~$\in [0, 1]$
                         & 0.01
                         & The threshold $\theta$ in the aggregation algorithm,
                            see (\ref{eq:strongly_coup}) in Section~\ref{sec:aggregation}.
                            See also the note at the bottom of this table. \\ \hline
%%\verb|mld_aggr_scale_| \par 
% \verb|AGGR_SCALE| & \verb|real(|\emph{kind\_parameter}\verb|)|
%                         & Any~real \par number~$\in [0, 1]$
%                         & 1.0
%                         & Scale factor applied to the threshold in going
%                           from level $ilev$ to level $ilev+1$. \\ \hline
%\verb|mld_aggr_omega_alg_|  \par 
%\verb|'AGGR_OMEGA_ALG'|& \verb|character(len=*)|
%                         & \texttt{'EIG\_EST'} \par \texttt{'USER\_CHOICE'}
%                         & \texttt{'EIG\_EST'}
%                         & How the damping parameter $\omega$ in the
%                           smoothed aggregation is obtained:
%                           either via an estimate of the spectral radius of
%                           $D^{-1}A$, where $A$ is the matrix at the current
%                           level and $D$ is the diagonal matrix with
%                           the same diagonal entires as $A$, or explicily
%                           specified by the user. \\ \hline
%\verb|mld_aggr_eig_|  \par 
%\verb|'AGGR_EIG'|    & \verb|character(len=*)|
%                        & \texttt{'A\_NORMI'}
%                         & \texttt{'A\_NORMI'}
%                         & How to estimate the spectral radius of $D^{-1}A$.
%                           Currently only the infinity norm estimate
%                           is available. \\ \hline
%\verb|mld_aggr_omega_val_| \par 
%\verb|'AGGR_OMEGA_VAL'|    & \verb|real(|\emph{kind\_parameter}\verb|)|
%                         & Any real \par number $>0$
%                         & $4/(3\rho(D^{-1}A))$
%                         & Damping parameter $\omega$ in the smoothed aggregation algorithm. 
%                           It must be set by the user if
%                           \verb|USER_CHOICE| was specified for 
%                           \verb|mld_aggr_omega_alg_|,
%                           otherwise it is computed by the library, using the
%                           selected estimate of the spectral radius $\rho(D^{-1}A)$ of
%                           $D^{-1}A$.\\ \hline
%\verb|mld_aggr_filter_| \par 
\verb|'AGGR_FILTER'|  
                         & \verb|character(len=*)|
                         & \texttt{'FILTER'} \par \texttt{'NOFILTER'}
                        & \texttt{'NOFILTER'} & Matrix used in computing the smoothed
                           prolongator: filtered or unfiltered (see~(\ref{eq:filtered}) in Section~\ref{sec:aggregation}). \\
\hline
\multicolumn{5}{|l|}{{\bfseries Note.} Different thresholds at different levels, such as
those used in \cite[Section~5.1]{VANEK_MANDEL_BREZINA}, can be easily set  by 
invoking the rou-} \\
\multicolumn{5}{|l|}{tine \texttt{set} with
the parameter \texttt{ilev}.} \\
\hline
\end{tabular}
\end{center}
\caption{Parameters defining the aggregation algorithm (continued).
\label{tab:p_aggregation_1}} 
\esideways
                     
\bsideways
\begin{center}
\begin{tabular}{|p{3.9cm}|l|p{1.7cm}|p{1.7cm}|p{8.6cm}|}
\hline
\verb|what|              & \textsc{data type}        &  \verb|val|      &  \textsc{default}  &
\textsc{comments} \\ \hline
%\multicolumn{5}{|c|}{\emph{coarse-space correction at the coarsest level}}\\ \hline
%\verb|mld_coarse_mat_|  \par 
\verb|'COARSE_MAT'|  & \verb|character(len=*)|
                         & \texttt{'DIST'} \par \texttt{'REPL'}
                         & \texttt{'REPL'}
                         & Coarsest matrix layout: distributed among the processes or
                           replicated on each of them. \\ \hline
%\verb|mld_coarse_solve_| \par 
\verb|'COARSE_SOLVE'| & \verb|character(len=*)|
                          & \texttt{'MUMPS'} \par \texttt{'UMF'} \par 
                           \texttt{'SLU'} \par \texttt{'SLUDIST'} \par 
                           \texttt{'JACOBI'} \par \texttt{'GS'} \par \texttt{'BJAC'} 
                         & See~Note.
                         & Solver used at the coarsest level: sequential
                           LU from MUMPS, UMFPACK, or SuperLU
                           (plus tri\-an\-gular solve); 
                           distributed LU from MUMPS or SuperLU\_Dist
                           (plus triangular solve);
                           point-Jacobi, hybrid Gauss-Seidel or block-Jacobi. \par
                           Note that \texttt{UMF} and \texttt{SLU} require the coarsest
                           matrix to be replicated, \texttt{SLUDIST},  \texttt{JACOBI},
                           \texttt{GS} and \texttt{BJAC} require it to be
                           distributed, and \texttt{MUMPS} can be used with either
                           a replicated or a distributed matrix. When any of the previous
                           solvers is specified, the matrix layout is set to a default
                           value which allows the use of the solver (see Remark 3, p.~24). 
                           Note also that UMFPACK and SuperLU\_Dist
                           are available only in double precision. \\ \hline
%\verb|mld_coarse_subsolve_| \par 
\verb|'COARSE_SUBSOLVE'| & \verb|character(len=*)|
                         & \texttt{'ILU'} \par \texttt{'ILUT'} \par \texttt{'MILU'} \par
                            \texttt{'MUMPS'} \par \texttt{'SLU'} \par \texttt{'UMF'} 
                         & See~Note.
                         & Solver for the diagonal blocks of the coarse matrix,
                           in case the block Jacobi solver
                           is chosen as coarsest-level solver: ILU($p$), ILU($p,t$),
                           MILU($p$), LU from MUMPS, SuperLU or UMFPACK 
                          (plus triangular solve).
                          Note that UMFPACK and SuperLU\_Dist
                          are available only in double precision. \\
\hline
\multicolumn{5}{|l|}{{\bfseries Note.} Defaults for \texttt{COARSE\_SOLVE} and 
\texttt{COARSE\_SUBSOLVE} are chosen in the following  order:} \\
\multicolumn{5}{|l|}{single precision version -- \texttt{MUMPS} if installed, 
                               then \texttt{SLU} if installed, 
                               \texttt{ILU} otherwise;}\\
\multicolumn{5}{|l|}{double precision version -- \texttt{UMF} if installed, 
                               then \texttt{MUMPS} if installed, then \texttt{SLU} if  
                               installed, \texttt{ILU} otherwise.}\\
\hline
\end{tabular}
\end{center}
\caption{Parameters defining the coarse-space correction at the coarsest
level.\label{tab:p_coarse}} 
\esideways

\bsideways
\begin{center}
\begin{tabular}{|p{3.9cm}|l|p{2cm}|p{1.5cm}|p{7.5cm}|}
\hline
\verb|what|              & \textsc{data type}        &  \verb|val|      &  \textsc{default}  &
\textsc{comments} \\ \hline
%\multicolumn{5}{|c|}{\emph{coarse-space correction at the coarsest level}}\\ \hline
%\verb|mld_coarse_sweeps_| \par 
\verb|'COARSE_SWEEPS'| & \verb|integer|                         
                         & Any integer \par number $> 0$
                         & 10
                         & Number of sweeps when \verb|JACOBI|, \verb|GS| or \verb|BJAC|
                           is chosen as coarsest-level solver. \\ \hline
%\verb|mld_coarse_fillin_| \par 
\verb|'COARSE_FILLIN'| & \verb|integer|
                         & Any integer \par number $\ge 0$
                         & 0
                         & Fill-in level $p$ of the ILU factorizations. \\ \hline
%\verb|mld_coarse_iluthrs_|  \par 
\verb|'COARSE_ILUTHRS'|
                         & \verb|real(|\emph{kind\_parameter}\verb|)|
                         & Any real \par number $\ge 0$
                         & 0
                         & Drop tolerance $t$ in the ILU($p,t$) factorization. \\
\hline
\end{tabular}
\end{center}
\caption{Parameters defining the coarse-space correction at the coarsest
level (continued).\label{tab:p_coarse_1}} 
\esideways

\bsideways
\begin{center}
\small
\begin{tabular}{|p{3.6cm}|l|p{1.9cm}|p{3.6cm}|p{6.5cm}|}
\hline
\verb|what|              & \textsc{data type}        &  \verb|val|      &  \textsc{default}  &
\textsc{comments} \\ \hline
%\multicolumn{5}{|c|}{\emph{basic one-level preconditioner (smoother)}} \\ \hline
%\verb|mld_smoother_type_|  \par 
\verb|'SMOOTHER_TYPE'|  & \verb|character(len=*)|
                         & \verb|'JACOBI'| \par \verb|'GS'| \par \verb|'BGS'| \par \verb|'BJAC'|
                            \par \verb|'AS'|
                         & \verb|'FBGS'|
                         & Type of smoother used in the multilevel preconditioner:
                            point-Jacobi, hybrid (forward) Gauss-Seidel,
                            hybrid backward Gauss-Seidel, block-Jacobi, and
                            Additive Schwarz. \par
                            It is ignored by one-level preconditioners. \\ \hline
%\verb|mld_sub_solve_| \par 
\verb|'SUB_SOLVE'|  & \verb|character(len=*)|
                         & \texttt{'JACOBI'} \par
                           \texttt{'GS'} \par \texttt{'BGS'} \par \texttt{'ILU'} \par 
                           \texttt{'ILUT'} \par \texttt{'MILU'} \par 
                           \par \texttt{'MUMPS'} \par \texttt{'SLU'} \par \texttt{'UMF'}
                         & \texttt{GS} and \texttt{BGS} for pre- and post-smoothers
                            of multilevel preconditioners, respectively \par
                            \texttt{ILU} for block-Jacobi and Additive Schwarz
                            one-level preconditioners
                         & The local solver to be used with the smoother or one-level
                            preconditioner (see Remark~2, page~24): point-Jacobi,
                            hybrid (forward) Gauss-Seidel, hybrid backward
                           Gauss-Seidel, ILU($p$),  ILU($p,t$), MILU($p$),
                           LU from MUMPS, SuperLU or UMFPACK
                           (plus triangular solve). See Note for details on hybrid
                           Gauss-Seidel. \\ \hline
%\verb|mld_moother_sweeps_| \par 
\verb|'SMOOTHER_SWEEPS'|  & \verb|integer|
                         & Any integer \par number~$\ge 0$
                         & 1
                         & Number of sweeps of the smoother or one-level preconditioner.
                            In the multilevel case, no pre-smother or
                            post-smoother is used if this parameter is set to 0 
                            together with \verb|pos='PRE'| or \verb|pos='POST|,
                           respectively. \\ \hline
%\verb|mld_sub_ovr_|  \par 
\verb|'SUB_OVR'|  & \verb|integer|
                         & Any integer \par number~$\ge 0$
                         & 1
                         & Number of overlap layers, for Additive Schwarz only. \\
\hline
\end{tabular}
\end{center}
\caption{Parameters defining the smoother or the details of the one-level preconditioner.
\label{tab:p_smoother}}  
\esideways

\bsideways
\begin{center}
\small
\begin{tabular}{|p{3cm}|l|p{2.5cm}|p{2.2cm}|p{7.1cm}|}
\hline
\verb|what|              & \textsc{data type}        &  \verb|val|      &  \textsc{default}  &
\textsc{comments} \\ \hline
%\verb|mld_sub_restr_|  \par 
\verb|'SUB_RESTR'|   & \verb|character(len=*)|
                         & \texttt{'HALO'} \par \texttt{'NONE'}
                         & \texttt{'HALO'}
                         & Type of restriction operator,  for Additive Schwarz only:
                           \texttt{HALO} for taking into account the overlap, \texttt{NONE} 
                           for neglecting it. \par
                           Note that \texttt{HALO} must be chosen for
                          the classical Addditive Schwarz smoother and its RAS variant.\\ \hline
%\verb|mld_sub_prol_| \par 
\verb|'SUB_PROL'|   & \verb|character(len=*)|
                         & \texttt{'SUM'} \par \texttt{'NONE'}
                         & \texttt{'NONE'}
                         & Type of prolongation operator, for Additive Schwarz only:
                           \texttt{SUM} for adding the contributions from the overlap, \texttt{NONE}
                           for neglecting them.   \par
                          Note that \texttt{SUM} must be chosen for the classical Additive
                          Schwarz smoother, and \texttt{NONE} for its RAS variant. \\ \hline
%\verb|mld_sub_fillin_| \par 
\verb|'SUB_FILLIN'|  & \verb|integer|
                         & Any integer \par number~$\ge 0$
                         & 0
                         & Fill-in level $p$ of the incomplete LU factorizations. \\ \hline
%\verb|mld_sub_iluthrs_| \par 
\verb|'SUB_ILUTHRS'|  & \verb|real(|\emph{kind\_parameter}\verb|)|
                         & Any real number~$\ge 0$
                         & 0
                         & Drop tolerance $t$ in the ILU($p,t$) factorization. \\ %\hline
\verb|'MUMPS_IPAR_ENTRY'|  & \verb|integer|
                         & Any integer  number
                         & 0
                         & Set an entry in the MUMPS integer control array, as
                           chosen via the \verb|idx| optional argument. \\ %\hline
\verb|'MUMPS_RPAR_ENTRY'|  & \verb|real|
                         & Any real number
                         & 0
                         & Set an entry in the MUMPS real control array, as
                           chosen via the \verb|idx| optional argument. \\ %\hline
%\verb|mld_sub_ren_| \par \verb|SUB_REN|   & \verb|character(len=*)|
%                         & \texttt{'RENUM\_NONE'}  \texttt{'RENUM\_GLOBAL'} %, \texttt{'RENUM_GPS'}
%                         & \texttt{'RENUM\_NONE'}
%                         & Row and column reordering of the local submatrices: no reordering,
%                           or reordering according to the global numbering of the rows and
%                           columns of the whole matrix. \\
% \verb|mld_solver_eps_| \par \verb|SOLVER_EPS|  & \verb|real|
%                          & Any~real number
%                          & 0
%                          & Stopping tolerance for iterative local solver
%                            (currently only Gauss-Seidel); if $\le0$, then
%                            perform prespecified number of iterations. \\ \hline
\hline
\end{tabular}
\end{center}
\caption{Parameters defining the smoother or the details of the one-level preconditioner
(continued).\label{tab:p_smoother_1}}  
\esideways


\clearpage

\subsection{Method hierarchy\_build\label{sec:hier_bld}}
  
\begin{center}
\verb|call p%hierarchy_build(a,desc_a,info)|\\
\end{center}

\noindent
This method builds the hierarchy of matrices and restriction/prolongation
operators for the multilevel preconditioner \verb|p|, according to the requirements
made by the user through the methods \verb|init| and \verb|set|.

{\vskip1.5\baselineskip\noindent\large\bfseries Arguments} \smallskip

\begin{tabular}{p{1.2cm}p{12cm}}
\verb|a|      & \verb|type(psb_|\emph{x}\verb|spmat_type), intent(in)|. \\
              & The sparse matrix structure containing the local part of the
                matrix to be preconditioned. Note that \emph{x} must be chosen according
                to the real/complex, 
single/double precision version of MLD2P4 under use.
                See the PSBLAS User's Guide for details \cite{PSBLASGUIDE}.\\
\verb|desc_a| & \verb|type(psb_desc_type), intent(in)|. \\
              & The communication descriptor of \verb|a|. See the PSBLAS User's Guide for
                details \cite{PSBLASGUIDE}.\\
%\verb|p|      & \verb|type(mld_|\emph{x}\verb|prec_type), intent(inout)|.\\
%              & The preconditioner data structure. Note that \emph{x} must be chosen according
%                to the real/complex, single/double precision version of MLD2P4 under use.\\
\verb|info|   & \verb|integer, intent(out)|.\\
              & Error code. If no error, 0 is returned. See Section~\ref{sec:errors} for details.\\
\end{tabular}


\clearpage

\subsection{Method smoothers\_build\label{sec:smooth_bld}}
 

\begin{center}
\verb|call p%smoothers_build(a,desc_a,p,info[,amold,vmold,imold])|\\
\end{center}

\noindent
This method builds the smoothers and the coarsest-level solvers for the
multilevel preconditioner \verb|p|, according to the requirements made by
the user through the methods \verb|init| and \verb|set|, and based on the aggregation
hierarchy produced by a previous call to \verb|hierarchy_build|
(see Section~\ref{sec:hier_bld}). 

{\vskip1.5\baselineskip\noindent\large\bfseries Arguments} \smallskip

\begin{tabular}{p{1.2cm}p{12cm}}
\verb|a|      & \verb|type(psb_|\emph{x}\verb|spmat_type), intent(in)|. \\
              & The sparse matrix structure containing the local part of the
                matrix to be preconditioned. Note that \emph{x} must be chosen according
                to the real/complex, single/double precision version of MLD2P4 under use.
                See the PSBLAS User's Guide for details \cite{PSBLASGUIDE}.\\
\verb|desc_a| & \verb|type(psb_desc_type), intent(in)|. \\
              & The communication descriptor of \verb|a|. See the PSBLAS User's Guide for
                details \cite{PSBLASGUIDE}.\\
%\verb|p|  & \verb|type(mld_|\emph{x}\verb|prec_type), intent(inout)|.\\
%               & The preconditioner data structure. Note that \emph{x} must be chosen according
%                  to the real/complex, single/double precision version of MLD2P4 under use.\\
\verb|info|   & \verb|integer, intent(out)|.\\
              & Error code. If no error, 0 is returned. See Section~\ref{sec:errors} for details.\\
\verb|amold| & \verb|class(psb_|\emph{x}\verb|_base_sparse_mat), intent(in), optional|. \\
 & The desired dynamic type for internal matrix
  components; this allows e.g. running on GPUs; it needs not be the
  same on all processes. See the PSBLAS User's Guide for
  details \cite{PSBLASGUIDE}. \\
  \verb|vmold| & \verb|class(psb_|\emph{x}\verb|_base_vect_type), intent(in), optional|. \\
  & The desired dynamic type for internal vector
  components; this allows e.g. running on GPUs. \\
  \verb|imold| & \verb|class(psb_i_base_vect_type), intent(in), optional|. \\
  & The desired dynamic type for internal integer vector
  components; this allows e.g. running on GPUs. \\
\end{tabular}

\clearpage

\subsection{Method build\label{sec:precbld}}
  
\begin{center}
\verb|call p%build(a,desc_a,info[,amold,vmold,imold])|\\
\end{center}

\noindent
This method builds the preconditioner \verb|p| according to the requirements
made by the user through the methods \verb|init| and \verb|set|
(see Sections~\ref{sec:hier_bld} and~\ref{sec:smooth_bld} for
multilevel preconditioners). It is mostly provided for backward
compatibility; indeed, it is internally implemented by invoking the
two previous methods \verb|hierarchy_build| and
\verb|smoothers_build|, whose nomenclature would however  be somewhat  
unnatural when dealing with simple one-level preconditioners. 

{\vskip1.5\baselineskip\noindent\large\bfseries Arguments} \smallskip

\begin{tabular}{p{1.2cm}p{12cm}}
\verb|a|  & \verb|type(psb_|\emph{x}\verb|spmat_type), intent(in)|. \\
              & The sparse matrix structure containing the local part of the
                matrix to be preconditioned. Note that \emph{x} must be chosen according
                to the real/complex, single/double precision version of MLD2P4 under use.
                See the PSBLAS User's Guide for details \cite{PSBLASGUIDE}.\\
\verb|desc_a| & \verb|type(psb_desc_type), intent(in)|. \\
              & The communication descriptor of \verb|a|. See the PSBLAS User's Guide for
                details \cite{PSBLASGUIDE}.\\
%\verb|p|  & \verb|type(mld_|\emph{x}\verb|prec_type), intent(inout)|.\\
%              & The preconditioner data structure. Note that \emph{x} must be chosen according
%                to the real/complex, single/double precision version of MLD2P4 under use.\\
\verb|info|   & \verb|integer, intent(out)|.\\
              & Error code. If no error, 0 is returned. See Section~\ref{sec:errors} for details.\\
\verb|amold| & \verb|class(psb_|\emph{x}\verb|_base_sparse_mat), intent(in), optional|. \\
 & The desired dynamic type for internal matrix
  components; this allows e.g. running on GPUs; it needs not be the
  same on all processes. See the PSBLAS User's Guide for
  details \cite{PSBLASGUIDE}. \\
  \verb|vmold| & \verb|class(psb_|\emph{x}\verb|_base_vect_type), intent(in), optional|. \\
  & The desired dynamic type for internal vector
  components; this allows e.g. running on GPUs. \\
  \verb|imold| & \verb|class(psb_i_base_vect_type), intent(in), optional|. \\
  & The desired dynamic type for internal integer vector
  components; this allows e.g. running on GPUs. \\
\end{tabular}

\vskip1.5\baselineskip
For compatibility with the previous versions of MLD2P4, this method can be also invoked
as follows:

\begin{center}
\verb|call mld_precbld(p,what,val,info[,amold,vmold,imold])|
\end{center}

\noindent
The method can be used to build multilevel preconditioners too.


\clearpage
\subsection{Method apply\label{sec:precapply}}

\begin{center}
\verb|call p%apply(x,y,desc_a,info [,trans,work])|\\
\end{center}

\noindent
This method computes $y = op(B^{-1})\, x$, where $B$ is a previously built
preconditioner, stored into \verb|p|, and $op$
denotes the preconditioner itself or its transpose, according to
the value of \verb|trans|.
Note that, when MLD2P4 is used with a Krylov solver from PSBLAS,
\verb|p%apply| is called within the PSBLAS method \verb|psb_krylov|
and hence it is completely transparent to the user.

{\vskip1.5\baselineskip\noindent\large\bfseries Arguments} \smallskip

\begin{tabular}{p{1.2cm}p{12cm}}
%\verb|p|      & \verb|type(mld_|\emph{x}\verb|prec_type), intent(inout)|.\\
%              & The preconditioner data structure, containing the local part of $M$.
%               Note that \emph{x} must be chosen according
%                to the real/complex, single/double precision version of MLD2P4 under use.\\
\verb|x|      & \emph{type}\verb|(|\emph{kind\_parameter}\verb|), dimension(:), intent(in)|.\\
              & The local part of the vector $x$. Note that \emph{type} and   
                \emph{kind\_parameter} must be chosen according
                to the real/complex, single/double precision version of MLD2P4 under use.\\
\verb|y|      & \emph{type}\verb|(|\emph{kind\_parameter}\verb|), dimension(:), intent(out)|.\\
              & The local part of the vector $y$. Note that \emph{type} and
                \emph{kind\_parameter} must be chosen according
                to the real/complex, single/double precision version of MLD2P4 under use.\\
\verb|desc_a| & \verb|type(psb_desc_type), intent(in)|. \\
              & The communication descriptor associated to the matrix to be
                preconditioned.\\
\verb|info|   & \verb|integer, intent(out)|.\\
              & Error code. If no error, 0 is returned. See Section~\ref{sec:errors} for details.\\
\verb|trans|  & \verb|character(len=1), optional, intent(in).|\\
              & If \verb|trans| = \verb|'N','n'| then $op(B^{-1}) = B^{-1}$;
                if \verb|trans| = \verb|'T','t'| then $op(B^{-1}) = B^{-T}$
                (transpose of $B^{-1})$;  if \verb|trans| = \verb|'C','c'| then $op(B^{-1}) = B^{-C}$
                (conjugate transpose of $B^{-1})$.\\
\verb|work|  & \emph{type}\verb|(|\emph{kind\_parameter}\verb|), dimension(:), optional, target|.\\
             & Workspace. Its size should be at
               least \verb|4 * psb_cd_get_local_| \verb|cols(desc_a)| (see the PSBLAS User's Guide).
               Note that \emph{type} and \emph{kind\_parameter} must be chosen according
               to the real/complex, single/double precision version of MLD2P4 under use.\\
\end{tabular}

\vskip1.5\baselineskip
For compatibility with the previous versions of MLD2P4, this method can be also invoked
as follows:

\begin{center}
\verb|call mld_precaply(p,what,val,info)|
\end{center}

\clearpage

\subsection{Method free\label{sec:precfree}}

\begin{center}
\verb|call p%free(p,info)|\\
\end{center}

\noindent
This method deallocates the preconditioner data structure \verb|p|.

{\vskip1.5\baselineskip\noindent\large\bfseries Arguments} \smallskip

\begin{tabular}{p{1.2cm}p{10.5cm}}
%\verb|p|      & \verb|type(mld_|\emph{x}\verb|prec_type), intent(inout)|.\\
%              & The preconditioner data structure. Note that \emph{x} must be chosen according
 %               to the real/complex, single/double precision version of MLD2P4 under use.\\
\verb|info|   & \verb|integer, intent(out)|.\\
              & Error code. If no error, 0 is returned. See Section~\ref{sec:errors} for details.\\
\end{tabular}

\vskip1.5\baselineskip
For compatibility with the previous versions of MLD2P4, this method can be also invoked
as follows:

\begin{center}
\verb|call mld_precfree(p,info)|
\end{center}


\clearpage

\subsection{Method descr\label{sec:precdescr}}

\begin{center}
\verb|call p%descr(info, [iout])|\\
\end{center}

\noindent
This method prints a description of the preconditioner \verb|p| to the standard output or
to a file. It must be called after \verb|hierachy_build| and \verb|smoothers_build|,
or \verb|build|, have been called.

{\vskip1.5\baselineskip\noindent\large\bfseries Arguments} \smallskip

\begin{tabular}{p{1.2cm}p{12cm}}
%\verb|p|      & \verb|type(mld_|\emph{x}\verb|prec_type), intent(in)|.\\
%              & The preconditioner data structure. Note that \emph{x} must be chosen according
%                to the real/complex, single/double precision version of MLD2P4 under use.\\
\verb|info|   & \verb|integer, intent(out)|.\\
              & Error code. If no error, 0 is returned. See Section~\ref{sec:errors} for details.\\
\verb|iout|   & \verb|integer, intent(in), optional|.\\
              & The id of the file where the preconditioner description
                will be printed; the default is the standard output.\\
\end{tabular}

\vskip1.5\baselineskip
For compatibility with the previous versions of MLD2P4, this method can be also invoked
as follows:

\begin{center}
\verb|call mld_precdescr(p,info [,iout])|
\end{center}


\subsection{Auxiliary Methods\label{sec:auxil}}
Various functionalities are implemented as additional methods of the
preconditioner object. 

\subsubsection{Method: dump}
  
\begin{center}
\verb|call p%dump(info[,istart,iend,prefix,head,ac,rp,smoother,solver,global_num])|\\
\end{center}

\noindent
Dump on file.

{\vskip1.5\baselineskip\noindent\large\bfseries Arguments} \smallskip

\begin{tabular}{p{1.2cm}p{12cm}}
%\verb|p|  & \verb|type(mld_|\emph{x}\verb|prec_type), intent(inout)|.\\
%              & The preconditioner data structure. Note that \emph{x} must be chosen according
%                to the real/complex, single/double precision version of MLD2P4 under use.\\
\verb|info|   & \verb|integer, intent(out)|.\\
              & Error code. If no error, 0 is returned. See Section~\ref{sec:errors} for details.\\
\verb|amold| & \verb|class(psb_|\emph{x}\verb|_base_sparse_mat), intent(in), optional|. \\
 & The desired dynamic type for internal matrix
  components; this allows e.g. running on GPUs; it needs not be the
  same on all processes. See the PSBLAS User's Guide for
  details \cite{PSBLASGUIDE}. \\
\end{tabular}


\subsubsection{Method: clone}
  
\begin{center}
\verb|call p%clone(pout,info)|\\
\end{center}

\noindent
Create a (deep) copy of the preconditioner object. 

{\vskip1.5\baselineskip\noindent\large\bfseries Arguments} \smallskip

\begin{tabular}{p{1.2cm}p{12cm}}
\verb|pout|  & \verb|type(mld_|\emph{x}\verb|prec_type), intent(out)|.\\
              & The copy of the preconditioner data structure. Note
                that \emph{x} must be chosen according 
                to the real/complex, single/double precision version of MLD2P4 under use.\\
\verb|info|   & \verb|integer, intent(out)|.\\
              & Error code. If no error, 0 is returned. See Section~\ref{sec:errors} for details.\\
\end{tabular}



\subsubsection{Method: sizeof}
  
\begin{center}
\verb|sz =  p%sizeof()|\\
\end{center}

\noindent
Return memory footprint in bytes. 

% {\vskip1.5\baselineskip\noindent\large\bfseries Arguments} \smallskip

% \begin{tabular}{p{1.2cm}p{12cm}}
% %\verb|p|  & \verb|type(mld_|\emph{x}\verb|prec_type), intent(inout)|.\\
% %              & The preconditioner data structure. Note that \emph{x} must be chosen according
% %                to the real/complex, single/double precision version of MLD2P4 under use.\\
% \verb|info|   & \verb|integer, intent(out)|.\\
%               & Error code. If no error, 0 is returned. See Section~\ref{sec:errors} for details.\\
% \end{tabular}

\subsubsection{Method: allocate\_wrk}
  
\begin{center}
\verb|call p%allocate_wrk(info[, vmold])|\\
\end{center}

\noindent
Allocate internal work vectors. Each application of the preconditioner
uses a number of  work vectors which are allocated internally as
necessary; therefore allocation and deallocation of memory occurs
multiple times during the execution of a Krylov method. In most cases
this strategy is perfectly acceptable, but 
on some platforms, most notably GPUs, memory allocation is
a slow  operation, and the default behaviour would lead to a
slowdown.  This method allows to trade space for time by preallocating 
the internal workspace outside of the invocation of a Krylov
method. When using GPUs or other specialized devices, the \verb|vmold|
argument is also necessary to ensure the internal work vectors are of
the appropriate dynamic type to exploit the accelerator hardware; when
allocation occurs internally this is taken care of based on the dynamic
type of the \verb|x| argument to the \verb|apply| method. 

{\vskip1.5\baselineskip\noindent\large\bfseries Arguments} \smallskip

\begin{tabular}{p{1.2cm}p{12cm}}
%\verb|p|  & \verb|type(mld_|\emph{x}\verb|prec_type), intent(inout)|.\\
%              & The preconditioner data structure. Note that \emph{x} must be chosen according
%                to the real/complex, single/double precision version of MLD2P4 under use.\\
\verb|info|   & \verb|integer, intent(out)|.\\
              & Error code. If no error, 0 is returned. See Section~\ref{sec:errors} for details.\\
  \verb|vmold| & \verb|class(psb_|\emph{x}\verb|_base_vect_type), intent(in), optional|. \\
  & The desired dynamic type for internal vector
  components; this allows e.g. running on GPUs. \\
\end{tabular}



\subsubsection{Method: free\_wrk}
  
\begin{center}
\verb|call p%free_wrk(info)|\\
\end{center}

\noindent
Deallocate internal work vectors. 

{\vskip1.5\baselineskip\noindent\large\bfseries Arguments} \smallskip

\begin{tabular}{p{1.2cm}p{12cm}}
%\verb|p|  & \verb|type(mld_|\emph{x}\verb|prec_type), intent(inout)|.\\
%              & The preconditioner data structure. Note that \emph{x} must be chosen according
%                to the real/complex, single/double precision version of MLD2P4 under use.\\
\verb|info|   & \verb|integer, intent(out)|.\\
              & Error code. If no error, 0 is returned. See Section~\ref{sec:errors} for details.\\
\end{tabular}





%%% Local Variables: 
%%% mode: latex
%%% TeX-master: "userguide"
%%% End: 
