\section{Getting Started\label{sec:started}}
\markboth{\textsc{AMG4PSBLAS User's and Reference Guide}}
         {\textsc{\ref{sec:started} Getting Started}}

This section  describes the basics for building and applying
AMG4PSBLAS one-level and multilevel (i.e., AMG) preconditioners with
the Krylov solvers included in PSBLAS~\cite{PSBLASGUIDE}.

The following steps are required:
\begin{enumerate}
\item \emph{Declare the preconditioner data structure}. It is a derived data type,
  \verb|amg_|\-\emph{x}\verb|prec_| \verb|type|, where \emph{x} may be
  \verb|s|, \verb|d|, \verb|c| or \verb|z|, according to the basic
  data type of the sparse matrix (\verb|s| = real single precision;
  \verb|d| = real double precision; 	\verb|c| = complex single
  precision; \verb|z| = complex double precision). 	This data
  structure is accessed by the user only through the AMG4PSBLAS
  routines, 	following an object-oriented approach.
\item \emph{Allocate and initialize the preconditioner data structure,
    according to a preconditioner type chosen by the user}. This is
  performed by the routine  	\fortinline|init|, which also sets
  defaults for each preconditioner 	type selected by the user. The
  preconditioner types and the defaults associated 	with them are
  given in Table~\ref{tab:precinit}, where the strings used by
  \fortinline|init| to identify the preconditioner types are also
  given. 	Note that these strings are valid also if uppercase
  letters are substituted by 	corresponding lowercase ones. 

\item \emph{Modify the selected preconditioner type, by properly
    setting   preconditioner parameters.} This is performed by the
  routine \fortinline|set|.   This routine must be called if the
  user wants to modify the default values  of the parameters
  associated with the selected preconditioner type, to obtain a
  variant   of that preconditioner. Examples of use of
  \fortinline|set| are given in   Section~\ref{sec:examples}; a
  complete list of all the   preconditioner parameters and their
  allowed and default values is provided in
  Section~\ref{sec:userinterface},
  Tables~\ref{tab:p_cycle}-\ref{tab:p_smoother_1}. 
\item \emph{Build the preconditioner for a given matrix}. If the selected preconditioner
 is multilevel, then two steps must be performed, as specified next.
\begin{enumerate}
\item[4.1] \emph{Build the AMG hierarchy for a given matrix.} This is
performed by the routine \fortinline|hierarchy_build|.
\item[4.2] \emph{Build the preconditioner for a given matrix.} This is performed
by the routine \fortinline|smoothers_build|.
\end{enumerate}
 If the selected preconditioner is one-level, it is built in a single step,
performed by the routine \fortinline|bld|.
\item \emph{Apply the preconditioner at each iteration of a Krylov solver.}
  This is performed by the method \fortinline|apply|. When using the PSBLAS Krylov solvers,
  this step is completely transparent to the user, since \fortinline|apply| is called
  by the PSBLAS routine implementing the Krylov solver (\fortinline|psb_krylov|).
\item \emph{Free the preconditioner data structure}. This is performed by
  the routine \fortinline|free|. This step is complementary to step 1 and should
  be performed when the preconditioner is no more used.
\end{enumerate}

All the previous routines are available as methods of the preconditioner object.
A detailed description of them is given in Section~\ref{sec:userinterface}.
Examples showing the basic use of AMG4PSBLAS are reported in Section~\ref{sec:examples}.

\begin{table}[h!]
\begin{center}
%{\small
\begin{tabular}{|l|p{2cm}|p{6.8cm}|}
\hline
\textsc{type}       & \textsc{string} & \textsc{default preconditioner} \\ \hline
No preconditioner &\fortinline|'NONE'|& Considered  to use the PSBLAS
                                    Krylov solvers with no preconditioner. \\ \hline
Diagonal          & \fortinline|'DIAG'|, \fortinline|'JACOBI'|, \fortinline|'L1-JACOBI'| & Diagonal preconditioner.
                         For any zero diagonal entry of the matrix to be preconditioned,
                         the corresponding entry of the preconditioner is set to~1.\\ \hline
Gauss-Seidel      & \fortinline|'GS'|, \fortinline|'L1-GS'|     & Hybrid Gauss-Seidel (forward), that is,
                                      global block Jacobi with
                                      Gauss-Seidel as local solver.\\ \hline
Symmetrized Gauss-Seidel      & \fortinline|'FBGS'|, \fortinline|'L1-FBGS'|     & Symmetrized hybrid Gauss-Seidel, that is,
                                      forward Gauss-Seidel followed by
                                                    backward Gauss-Seidel.\\ \hline
Block Jacobi      & \fortinline|'BJAC'|, \fortinline|'L1-BJAC'| & Block-Jacobi with ILU(0) on the local blocks.\\ \hline
Additive Schwarz  & \fortinline|'AS'|   & Additive Schwarz (AS),
                                    with overlap~1 and ILU(0) on the local blocks. \\ \hline
Multilevel        &\fortinline|'ML'|    & V-cycle with one hybrid forward Gauss-Seidel
                                    (GS) sweep as pre-smoother and one hybrid backward
                                    GS sweep as post-smoother, decoupled smoothed aggregation
                                   as coarsening algorithm, and LU (plus triangular solve)
                                   as coarsest-level solver. See the default values in
                                   Tables~\ref{tab:p_cycle}-\ref{tab:p_smoother_1}
                                   for further details of the preconditioner. \\
\hline
\end{tabular}
%}
\caption{Preconditioner types, corresponding strings and default choices.
\label{tab:precinit}}
\end{center}
\end{table}

Note that the module \fortinline|amg_prec_mod|, containing the definition of the
preconditioner data type and the interfaces to the routines of AMG4PSBLAS,
must be used in any program calling such routines.
The modules \fortinline|psb_base_mod|, for the sparse matrix and communication descriptor
data types, and \fortinline|psb_krylov_mod|, for interfacing with the
Krylov solvers, must be also used (see Section~\ref{sec:examples}). \\

\textbf{Remark 1.} Coarsest-level solvers based on the LU factorization,
such as those implemented in UMFPACK, MUMPS, SuperLU, and SuperLU\_Dist,
usually lead to smaller numbers of preconditioned Krylov
iterations than inexact solvers, when the linear system comes from
a standard discretization of basic scalar elliptic PDE problems. However,
this does not necessarily correspond to the shortest execution time
on parallel~computers.


\subsection{Examples\label{sec:examples}}

The code reported in Figure~\ref{fig:ex1} shows how to set and apply the default
multilevel preconditioner available in the real double precision version
of AMG4PSBLAS (see Table~\ref{tab:precinit}). This preconditioner is chosen
by simply specifying \fortinline|'ML'| as the second argument of \fortinline|P%init|
(a call to \fortinline|P%set| is not needed) and is applied with the CG
solver provided by PSBLAS (the matrix of the system to be solved is
assumed to be positive definite). As previously observed, the modules
\fortinline|psb_base_mod|, \fortinline|amg_prec_mod| and \fortinline|psb_krylov_mod|
must be used by the example program.

The part of the code dealing with reading and assembling  the sparse
matrix and the right-hand side vector and the deallocation of the
relevant data structures, performed 
through the PSBLAS routines for sparse matrix and vector management, is not reported
here for the sake of conciseness.
The complete code can be found in the example program file \verb|amg_dexample_ml.f90|,
in the directory \verb|samples/simple/fileread| of the AMG4PSBLAS implementation (see
Section~\ref{sec:ex_and_test}). A sample test problem along with the relevant
input data is available in \verb|samples/simple/fileread/runs|.
For details on the use of the PSBLAS routines, see the PSBLAS User's
Guide~\cite{PSBLASGUIDE}.

The setup and application of the default multilevel preconditioner
for the real single precision and the complex, single and double
precision, versions are obtained with straightforward modifications of the previous
example (see Section~\ref{sec:userinterface} for details). If these versions are installed,
the corresponding codes are available in \verb|samples/simple/fileread/|.

\begin{listing}[tbp]
\begin{center}
\begin{minipage}{.90\textwidth}
\ifpdf
\begin{minted}[breaklines=true,bgcolor=bg,fontsize=\small]{fortran}
  use psb_base_mod
  use amg_prec_mod
  use psb_krylov_mod
... ...
!
! sparse matrix
  type(psb_dspmat_type) :: A
! sparse matrix descriptor
  type(psb_desc_type)   :: desc_A
! preconditioner
  type(amg_dprec_type)  :: P
! right-hand side and solution vectors
  type(psb_d_vect_type) :: b, x
... ...
!
! initialize the parallel environment
  call psb_init(ctxt)
  call psb_info(ctxt,iam,np)
... ...
!
! read and assemble the spd matrix A and the right-hand side b
! using PSBLAS routines for sparse matrix / vector management
... ...
!
! initialize the default multilevel preconditioner, i.e. V-cycle
! with basic smoothed aggregation, 1 hybrid forward/backward
! GS sweep as pre/post-smoother and UMFPACK as coarsest-level
! solver
  call P%init('ML',info)
!
! build the preconditioner
  call P%hierarchy_build(A,desc_A,info)
  call P%smoothers_build(A,desc_A,info)

!
! set the solver parameters and the initial guess
... ...
!
! solve Ax=b with preconditioned CG
  call psb_krylov('CG',A,P,b,x,tol,desc_A,info)
... ...
!
! deallocate the preconditioner
  call P%free(info)
!
! deallocate other data structures
... ...
!
! exit the parallel environment
  call psb_exit(ctxt)
stop
\end{minted}
\else
{\small
\begin{verbatim}
  use psb_base_mod
  use amg_prec_mod
  use psb_krylov_mod
... ...
!
! sparse matrix
  type(psb_dspmat_type) :: A
! sparse matrix descriptor
  type(psb_desc_type)   :: desc_A
! preconditioner
  type(amg_dprec_type)  :: P
! right-hand side and solution vectors
  type(psb_d_vect_type) :: b, x
... ...
!
! initialize the parallel environment
  call psb_init(ctxt)
  call psb_info(ctxt,iam,np)
... ...
!
! read and assemble the spd matrix A and the right-hand side b
! using PSBLAS routines for sparse matrix / vector management
... ...
!
! initialize the default multilevel preconditioner, i.e. V-cycle
! with basic smoothed aggregation, 1 hybrid forward/backward
! GS sweep as pre/post-smoother and UMFPACK as coarsest-level
! solver
  call P%init('ML',info)
!
! build the preconditioner
  call P%hierarchy_build(A,desc_A,info)
  call P%smoothers_build(A,desc_A,info)

!
! set the solver parameters and the initial guess
  ... ...
!
! solve Ax=b with preconditioned CG
  call psb_krylov('CG',A,P,b,x,tol,desc_A,info)
  ... ...
!
! deallocate the preconditioner
  call P%free(info)
!
! deallocate other data structures
  ... ...
!
! exit the parallel environment
  call psb_exit(ctxt)
  stop
\end{verbatim}
}
\fi
\end{minipage}
\caption{setup and application of the default multilevel preconditioner (example 1).
\label{fig:ex1}}
\end{center}
\end{listing}

Different versions of the multilevel preconditioner can be obtained by changing
the default values of the preconditioner parameters. The code reported in
Figure~\ref{fig:ex2} shows how to set a V-cycle preconditioner
which applies 1 block-Jacobi sweep as pre- and post-smoother,
and solves the coarsest-level system with 8 block-Jacobi sweeps.
Note that the ILU(0) factorization (plus triangular solve) is used as
local solver for the block-Jacobi sweeps, since this is the default associated
with block-Jacobi and set by~\fortinline|P%init|.
Furthermore, specifying block-Jacobi as coarsest-level
solver implies that the coarsest-level matrix is distributed
among the processes.
Figure~\ref{fig:ex3} shows how to set a W-cycle preconditioner using the Coarsening based on Compatible Weighted Matching, aggregates of size at most $8$ and smoothed prolongators. It applies
2 hybrid Gauss-Seidel sweeps as pre- and post-smoother,
and solves the coarsest-level system with the parallel flexible Conjugate Gradient method (KRM) coupled with the block-Jacobi preconditioner having ILU(0) on the blocks. Default parameters are used for stopping criterion of the coarsest solver.
Note that, also in this case, specifying KRM as coarsest-level
solver implies that the coarsest-level matrix is distributed
among the processes.
%It is specified that the coarsest-level

%matrix is distributed, since MUMPS can be used on both

%replicated and distributed matrices, and by default

%it is used on replicated ones.
%Note the use of the parameter \fortinline|pos|
%to specify a property only for the pre-smoother or the post-smoother
%(see Section~\ref{sec:precset} for more details).
The code fragments shown in Figures~\ref{fig:ex2} and \ref{fig:ex3} are
included in the example program file \verb|amg_dexample_ml.f90| too.

Finally, Figure~\ref{fig:ex4} shows the setup of a one-level
additive Schwarz preconditioner, i.e., RAS with overlap 2.
Note also that a Krylov method different from CG must be used to solve
the preconditioned system, since the preconditione in nonsymmetric.
The corresponding example program is available in the file
\verb|amg_dexample_1lev.f90|.

For all the previous preconditioners, example programs where the sparse matrix and
the right-hand side are generated by discretizing a PDE with Dirichlet
boundary conditions are also available in the directory \verb|samples/simple/pdegen|.
\vspace{-1em}\begin{listing}[tbh]
\ifpdf%
\begin{minted}[breaklines=true,bgcolor=bg,fontsize=\small]{fortran}
! build a V-cycle preconditioner with 1 block-Jacobi sweep (with
! ILU(0) on the blocks) as pre- and post-smoother, and 8  block-Jacobi
! sweeps (with ILU(0) on the blocks) as coarsest-level solver
call P%init('ML',info)
call P%set('SMOOTHER_TYPE','BJAC',info)
call P%set('COARSE_SOLVE','BJAC',info)
call P%set('COARSE_SWEEPS',8,info)
call P%hierarchy_build(A,desc_A,info)
call P%smoothers_build(A,desc_A,info)
\end{minted}
\else%
\begin{center}
\begin{minipage}{.90\textwidth}
{\small
\begin{verbatim}
... ...
! build a V-cycle preconditioner with 1 block-Jacobi sweep (with
! ILU(0) on the blocks) as pre- and post-smoother, and 8  block-Jacobi
! sweeps (with ILU(0) on the blocks) as coarsest-level solver
  call P%init('ML',info)
  call P%set('SMOOTHER_TYPE','BJAC',info)
  call P%set('COARSE_SOLVE','BJAC',info)
  call P%set('COARSE_SWEEPS',8,info)
  call P%hierarchy_build(A,desc_A,info)
  call P%smoothers_build(A,desc_A,info)
... ...
\end{verbatim}
}
\end{minipage}
\end{center}
\fi\vspace{-2em}%
\caption{setup of a multilevel preconditioner based on the default decoupled coarsening\label{fig:ex2}}
\end{listing}\vspace*{-2em}
\begin{listing}[h!]
\ifpdf
\begin{minted}[breaklines=true,bgcolor=bg,fontsize=\small]{fortran}
!build a W-cycle using the coupled coarsening based on weighted matching, 
!aggregates of size at most 8 and smoothed prolongators,
!2 hybrid Gauss-Seidel sweeps as pre- and post-smoother,
!and parallel flexible Conjugate Gradient coupled with the block-Jacobi 
!preconditioner having ILU(0) on the blocks as coarsest solver. 
call P%init('ML',info)
call P%set('PAR_AGGR_ALG','COUPLED',info)
call P%set('AGGR_TYPE','MATCHBOXP',info)
call P%set('AGGR_SIZE',8,info)
call P%set('ML_CYCLE','WCYCLE',info)
call P%set('SMOOTHER_TYPE','FBGS',info)
call P%set('SMOOTHER_SWEEPS',2,info)
call P%set('COARSE_SOLVE','KRM',info)
call P%hierarchy_build(A,desc_A,info)
call P%smoothers_build(A,desc_A,info)
\end{minted}
\else
\begin{center}
\begin{minipage}{.90\textwidth}
{\small
\begin{verbatim}
... ...
! build a W-cycle preconditioner with 2 hybrid Gauss-Seidel sweeps
! as pre- and post-smoother, a distributed coarsest
! matrix, and MUMPS as coarsest-level solver
  call P%init('ML',info)
  call P%set('PAR_AGGR_ALG','COUPLED',info)
call P%set('AGGR_TYPE','MATCHBOXP',info)
call P%set('AGGR_SIZE',8,info)
  call P%set('ML_CYCLE','WCYCLE',info)
  call P%set('SMOOTHER_TYPE','FBGS',info)
  call P%set('SMOOTHER_SWEEPS',2,info)
  call P%set('COARSE_SOLVE','KRM',info)
  call P%hierarchy_build(A,desc_A,info)
  call P%smoothers_build(A,desc_A,info)
... ...
\end{verbatim}
}
\end{minipage}
\end{center}
\fi\vspace{-2em}%
\caption{setup of a multilevel preconditioner based on the coupled coarsening using weighted matching\label{fig:ex3}}
\end{listing}\vspace*{-2em}
\begin{listing}[h!]
\ifpdf
\begin{minted}[breaklines=true,bgcolor=bg,fontsize=\small]{fortran}
! build a one-level RAS with overlap 2 and ILU(0) on the local blocks.
call P%init('AS',info)
call P%set('SUB_OVR',2,info)
call P%build(A,desc_A,info)
... ...
! solve Ax=b with preconditioned BiCGSTAB
  call psb_krylov('BICGSTAB',A,P,b,x,tol,desc_A,info)
\end{minted}
\else
\begin{center}
\begin{minipage}{.90\textwidth}
{\small
\begin{verbatim}
... ...
! set RAS with overlap 2 and ILU(0) on the local blocks
  call P%init('AS',info)
  call P%set('SUB_OVR',2,info)
  call P%bld(A,desc_A,info)
... ...
! solve Ax=b with preconditioned BiCGSTAB
  call psb_krylov('BICGSTAB',A,P,b,x,tol,desc_A,info)
\end{verbatim}
}
\end{minipage}
\end{center}
\fi\vspace{-2em}%
\caption{setup of a one-level Schwarz preconditioner.\label{fig:ex4}}
\end{listing}



\subsection{GPU example\label{sec:gpu-example}}

The code discussed here  shows how to set  up a
program exploiting the combined GPU capabilities of PSBLAS and
AMG4PSBLAS. The code example is available in the source distribution
directory \verb|amg4psblas/examples/gpu|. 

First of all, we need to include the appropriate modules and
declare some auxiliary variables:
\begin{listing}[h!]
\ifpdf
\begin{minted}[breaklines=true,bgcolor=bg,fontsize=\small]{fortran}
program amg_dexample_gpu
  use psb_base_mod
  use amg_prec_mod
  use psb_krylov_mod
  use psb_util_mod
  use psb_gpu_mod
  use data_input
  use amg_d_pde_mod
  implicit none
  .......
  ! GPU variables
  type(psb_d_hlg_sparse_mat) :: agmold
  type(psb_d_vect_gpu)       :: vgmold
  type(psb_i_vect_gpu)       :: igmold
\end{minted}
\else
\begin{center}
\begin{minipage}{.90\textwidth}
{\small
\begin{verbatim}
program amg_dexample_gpu
  use psb_base_mod
  use amg_prec_mod
  use psb_krylov_mod
  use psb_util_mod
  use psb_gpu_mod
  use data_input
  use amg_d_pde_mod
  implicit none
  .......
  ! GPU variables
  type(psb_d_hlg_sparse_mat) :: agmold
  type(psb_d_vect_gpu)       :: vgmold
  type(psb_i_vect_gpu)       :: igmold

 \end{verbatim}
}
\end{minipage}
\end{center}
\fi
\caption{setup of a GPU-enabled test program part one.\label{fig:gpu-ex1}}
\end{listing}
In this particular example we are choosing to employ a \verb|HLG| data
structure for sparse matrices on GPUs; for more information please
refer to the PSBLAS-EXT users' guide.

We then have to initialize the GPU environment, and pass the
appropriate MOLD variables to the build methods (see also the PSBLAS
and PSBLAS-EXT users' guides).
\begin{listing}[h!]
\ifpdf
\begin{minted}[breaklines=true,bgcolor=bg,fontsize=\small]{fortran}
  call psb_init(ctxt)
  call psb_info(ctxt,iam,np)
  !
  ! BEWARE: if you have NGPUS  per node, the default is to
  ! attach to mod(IAM,NGPUS)
  !
  call psb_gpu_init(ictxt)
  ......
  t1 = psb_wtime()
  call prec%smoothers_build(a,desc_a,info, amold=agmold, vmold=vgmold, imold=igmold)

\end{minted}
\else
\begin{center}
\begin{minipage}{.90\textwidth}
{\small
\begin{verbatim}
  call psb_init(ctxt)
  call psb_info(ctxt,iam,np)
  !
  ! BEWARE: if you have NGPUS  per node, the default is to
  ! attach to mod(IAM,NGPUS)
  !
  call psb_gpu_init(ictxt)
  ......
  t1 = psb_wtime()
  call prec%smoothers_build(a,desc_a,info, amold=agmold, vmold=vgmold, imold=igmold)

 \end{verbatim}
}
\end{minipage}
\end{center}
\fi
\caption{setup of a GPU-enabled test program part two.\label{fig:gpu-ex2}}
\end{listing}
Finally, we convert the input matrix, the descriptor and the vectors
to use a GPU-enabled internal storage format. 
We then preallocate the preconditioner workspace before entering the
Krylov method. At the end of the code, we close the GPU environment 
\begin{listing}[h!]
\ifpdf
\begin{minted}[breaklines=true,bgcolor=bg,fontsize=\small]{fortran}
  call desc_a%cnv(mold=igmold)
  call a%cscnv(info,mold=agmold)
  call psb_geasb(x,desc_a,info,mold=vgmold)
  call psb_geasb(b,desc_a,info,mold=vgmold)

  !
  ! iterative method parameters
  !
  call psb_barrier(ctxt)
  call prec%allocate_wrk(info)
  t1 = psb_wtime()
  call psb_krylov(s_choice%kmethd,a,prec,b,x,s_choice%eps,&
       & desc_a,info,itmax=s_choice%itmax,iter=iter,err=err,itrace=s_choice%itrace,&
       & istop=s_choice%istopc,irst=s_choice%irst)
  call prec%deallocate_wrk(info)
  call psb_barrier(ctxt)
  tslv = psb_wtime() - t1

  ......
  call psb_gpu_exit()
  call psb_exit(ctxt)
  stop
  

\end{minted}
\else
\begin{center}
\begin{minipage}{.90\textwidth}
{\small
\begin{verbatim}
  call desc_a%cnv(mold=igmold)
  call a%cscnv(info,mold=agmold)
  call psb_geasb(x,desc_a,info,mold=vgmold)
  call psb_geasb(b,desc_a,info,mold=vgmold)

  !
  ! iterative method parameters
  !
  call psb_barrier(ctxt)
  call prec%allocate_wrk(info)
  t1 = psb_wtime()
  call psb_krylov(s_choice%kmethd,a,prec,b,x,s_choice%eps,&
       & desc_a,info,itmax=s_choice%itmax,iter=iter,err=err,itrace=s_choice%itrace,&
       & istop=s_choice%istopc,irst=s_choice%irst)
  call prec%deallocate_wrk(info)
  call psb_barrier(ctxt)
  tslv = psb_wtime() - t1

  ......
  call psb_gpu_exit()
  call psb_exit(ctxt)
  stop
  
 \end{verbatim}
}
\end{minipage}
\end{center}
\fi
\caption{setup of a GPU-enabled test program part three.\label{fig:gpu-ex3}}
\end{listing}

It is very important to employ solvers that are suited
to the GPU, i.e.  solvers that do NOT  employ triangular
system solve kernels.  Solvers that satisfy this constraint include:
\begin{itemize}
\item \verb|JACOBI|
\item \verb|INVK|
\item \verb|INVT|
\item \verb|AINV|
\end{itemize}
and their $\ell_1$ variants. 

%%% Local Variables:
%%% mode: latex
%%% TeX-master: "userguide"
%%% End:
