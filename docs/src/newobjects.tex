
\clearpage

\section{Adding new smoother and solver objects to AMG4PSBLAS\label{sec:adding}}

Developers can add completely new smoother and/or solver classes
derived from the base objects in the library (see Remark~2 in Section~\ref{sec:precset}),
without recompiling the library itself.

To do so, it is necessary first to select the base type to be extended.
In our experience, it is quite likely that the new application needs
only the definition of a ``solver'' object, which is almost
always acting only on the local part of the distributed matrix.
The parallel actions required to connect the various solver objects
are most often already provided by the block-Jacobi or the additive
Schwarz smoothers.  To define a new solver, the developer will then
have to define its components and methods, perhaps taking one of the
predefined solvers as a starting point, if possible.

Once the new smoother/solver class has been developed, to use it in
the context of the multilevel preconditioners it is necessary to:
\begin{itemize}
\item declare in the application program a variable of the new type;
\item  pass that variable as the argument to the \verb|set| routine as in the
following:
\begin{center}
\verb|call p%set(smoother,info [,ilev,ilmax,pos])|\\
\verb|call p%set(solver,info [,ilev,ilmax,pos])|
\end{center}
\item link the code implementing the various methods into the application executable.
\end{itemize}
The new solver object is then dynamically included in the
preconditioner structure, and acts as a \emph{mold} to which the
preconditioner will conform, even though the AMG4PSBLAS library has not
been modified to account for this new development.

It is possible to define new values for the keyword \verb|WHAT| in the
\verb|set| routine; if the library code does not recognize a keyword,
it passes it down the composition hierarchy (levels containing
smoothers containing in turn solvers), so that it can be eventually caught by
the new solver. By the same token, any keyword/value pair that does not pertain to
a given smoother should be passed down to the contained solver, and
any keyword/value pair that does not pertain to a given solver is by
default ignored.

An example is provided in the source code distribution under the
folder \verb|tests/newslv|. In this example we are implementing a new
incomplete factorization variant (which is simply the ILU(0)
factorization under a new name). Because of the specifics of  this case, it is
possible to reuse the basic structure of the ILU solver, with its
L/D/U components and the methods needed to apply the solver; only a
few methods, such as the description and most importantly the build,
need to be ovverridden (rewritten).

The interfaces for the calls shown above are defined using
\begin{center}
\begin{tabular}{p{1.4cm}p{12cm}}
\verb|smoother| & \verb|class(amg_x_base_smoother_type)| \\
              & The user-defined new smoother to be employed in the
                preconditioner.\\
\verb|solver| & \verb|class(amg_x_base_solver_type)| \\
              & The user-defined new solver to be employed in the
                preconditioner.
\end{tabular}
\end{center}
The other arguments are defined in the way described in
Sec.~\ref{sec:precset}.  As an example, in  the \verb|tests/newslv|
code we define a new object of type \verb|amg_d_tlu_solver_type|, and
we pass it as follows:
\begin{verbatim}

  ! sparse matrix and preconditioner
  type(psb_dspmat_type) :: a
  type(amg_dprec_type)  :: prec
  type(amg_d_tlu_solver_type) :: tlusv

......
  !
  !  prepare the preconditioner: an ML with defaults, but with TLU solver at
  !  intermediate levels. All other parameters are at default values.
  !
  call prec%init('ML',       info)
  call prec%hierarchy_build(a,desc_a,info)
  nlv = prec%get_nlevs()
  call prec%set(tlusv,   info,ilev=1,ilmax=max(1,nlv-1))
  call prec%smoothers_build(a,desc_a,info)

\end{verbatim}
