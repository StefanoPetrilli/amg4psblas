\section{General Overview\label{sec:overview}}
\markboth{\textsc{AMG4PSBLAS User's and Reference Guide}}
         {\textsc{\ref{sec:overview} General Overview}}

The \textsc{Algebraic MultiGrid Preconditioners Package based on
PSBLAS} (\textsc{AMG\-4\-PSBLAS}) provides parallel Algebraic MultiGrid (AMG) preconditioners (see, e.g., \cite{Briggs2000,Stuben_01}),
to be used in the iterative solution of  linear systems,
\begin{equation}
Ax=b,
\label{system1}
\end{equation}
where $A$ is a square, real or complex, sparse symmetric positive definite (s.p.d) matrix.
%
%\textbf{NOTA: Caso non simmetrico, aggregazione con $(A+A^T)$ fatta!
%Dovremmo implementare uno smoothed prolongator
%adeguato e fare qualcosa di consistente anche con 1-lev Schwarz.}
%

The preconditioners implemented in AMG4PSBLAS are obtained by combining
3 different types of AMG cycles with smoothers and coarsest-level solvers. The V-, W-, and a version of a Krylov-type cycle (K-cycle)~\cite{Briggs2000,Notay2008} are available, which can be combined with Jacobi hybrid
%\footnote{see Note 2 in Table~\ref{tab:p_coarse}, p.~28.}
forward/backward Gauss-Seidel, block-Jacobi, and additive Schwarz smoothers. Also $\ell_1$ versions of Jacobi, block-Jacobi and Gauss-Seidel smoothers are available.
An algebraic approach is used to generate a hierarchy of
coarse-level matrices and operators, without explicitly using any information on the
geometry of the original problem, e.g., the discretization of a PDE. To this end,
two different coarsening strategies, based on aggregation, are available:
\begin{itemize}
\item a decoupled version of the well known smoothed aggregation procedure proposed in~\cite{BREZINA_VANEK,VANEK_MANDEL_BREZINA}, and already included in the previous versions of the package~\cite{BDDF2007,MLD2P4_TOMS};
\item the first parallel implementation of a coupled version of Coarsening based on Compatible Weighted Matching introduced in~\cite{DV2013,DFV2018} and described in details in~\cite{DDF2020};
\end{itemize}

Either exact or approximate solvers can be used on the coarsest-level system. Specifically, different sparse LU factorizations from external
packages, native incomplete LU and approximate inverse factorizations, weighted Jacobi, hybrid Gauss-Seidel, block-Jacobi solvers and recursive call to preconditioned Krylov methods are available. All the smoothers can be also exploited as one-level
preconditioners.

AMG4PSBLAS is written in Fortran~2003, following an
object-oriented design through the exploitation of features
such as abstract data type creation, type extension, functional overloading, and
dynamic memory management.
The parallel implementation is based on a Single Program Multiple Data
(SPMD) paradigm.  Single and
double precision implementations of AMG4PSBLAS are available for both the
real and the complex case, which can be used through a single
interface.

AMG4PSBLAS has been designed to implement scalable and easy-to-use
multilevel preconditioners in the context of the PSBLAS (Parallel Sparse BLAS)
computational framework~\cite{psblas_00,PSBLAS3}. PSBLAS provides basic linear algebra
operators and data management facilities for distributed sparse matrices,
kernels for sequential incomplete factorizations needed for the parallel block-Jacobi and additive Schwarz smoothers, and
parallel Krylov solvers which can be used with the AMG4PSBLAS preconditioners.
The choice of PSBLAS has been mainly motivated by the need of having
a portable and efficient software infrastructure implementing ``de facto'' standard
parallel sparse linear algebra kernels, to pursue goals such as performance,
portability, modularity ed extensibility in the development of the preconditioner
package. On the other hand, the implementation of AMG4PSBLAS, which was driven by the need to face the exascale challenge, has led to some important  revisions and extentions of the PSBLAS infrastructure.
The inter-process comunication required by AMG4PSBLAS is encapsulated
in the PSBLAS routines;
therefore, AMG4PSBLAS can be run on any parallel machine where PSBLAS
implementations are available. In the most recent version of PSBLAS (release 3.7), a plug-in for GPU is included; it includes CUDA versions of
main vector operations and of sparse matrix-vector multiplication, so that Krylov methods coupled with AMG4PBLAS preconditioners 
relying on Jacobi and block-Jacobi smoothers with sparse approximate inverses on the blocks can be efficiently executed on cluster of GPUs.

AMG4PSBLAS has a layered and modular software architecture where three main layers can be
identified.  The lower layer consists of the PSBLAS kernels, the middle one implements
the construction and application phases of the preconditioners, and the upper one
provides a uniform interface to all the preconditioners.
This architecture allows for different levels of use of the package:
few black-box routines at the upper layer allow all users to easily
build and apply any preconditioner available in AMG4PSBLAS;
facilities are also available allowing expert users to extend the set of smoothers
and solvers for building new versions of the preconditioners (see
Section~\ref{sec:adding}).

This guide is organized as follows. General information on the distribution of the source
code is reported in Section~\ref{sec:distribution}, while details on the configuration
and installation of the package are given in Section~\ref{sec:building}. The basics for building and applying the
preconditioners with the Krylov solvers implemented in PSBLAS are reported
in~Section~\ref{sec:started}, where the Fortran codes of a few sample programs
are also shown. A reference guide for the user interface routines is provided
in Section~\ref{sec:userinterface}. Information on the extension of the package
through the addition of new smoothers and solvers is reported in Section~\ref{sec:adding}.
The error handling mechanism used by the package
is briefly described in Section~\ref{sec:errors}. The copyright terms concerning the
distribution and modification of AMG4PSBLAS are reported in Appendix~\ref{sec:license}.

%%% Local Variables:
%%% mode: latex
%%% TeX-master: "userguide"
%%% End:
