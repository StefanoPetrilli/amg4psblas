\section*{Abstract}
\ifpdf
\addcontentsline{toc}{section}{Abstract}
\fi

\textsc{AMG4PSBLAS (Algebraic MultiGrid Preconditioners Package
based on PSBLAS}) is a package of parallel algebraic multilevel preconditioners included in the PSCToolkit (Parallel Sparse Computation Toolkit) software framework.
It is a progress of a software development project started in 2007, named MLD2P4, which originally implemented a 
multilevel version of some domain decomposition preconditioners of additive-Schwarz type and was based on a parallel decoupled version of the well known smoothed
aggregation method to generate the multilevel hierarchy of coarser matrices. 
In the last years, within the context of the EU-H2020 EoCoE project (Energy Oriented Center of Excellence), the package is being extended for including new algorithms and 
functionalities to setup and apply new AMG preconditioners with the final aims of improving efficiency and scalability when tens of thousands cores are
used and of boosting reliability in dealing with general symmetric positive definite linear systems. 
Due to the significant number of changes and the increase in scope, we decided to rename the package as AMG4PSBLAS.

AMG4PSBLAS is designed to provide scalable and easy-to-use preconditioners
in the context of the PSBLAS (Parallel Sparse Basic Linear Algebra Subprograms)
computational framework and can be used in conjuction with the Krylov solvers
available in this framework.
Our package is based on a completely algebraic approach and users level interfaces
assume that the system matrix and preconditioners are represented as PSBLAS
distributed sparse matrices.
AMG4PSBLAS enables the user to easily specify different
features of an algebraic multilevel preconditioner, thus allowing to experiment
with different preconditioners for the problem and parallel computers at hand.

The package employs object-oriented design techniques in
Fortran~2003, with interfaces to additional third party libraries
such as MUMPS, UMFPACK, SuperLU, and SuperLU\_Dist, which
can be exploited in building multilevel preconditioners. The parallel
implementation is based on a Single Program Multiple Data (SPMD)
paradigm; the inter-process communication is based on MPI and
is managed mainly through PSBLAS.

This guide provides a brief description of the functionalities and
the user interface of AMG4PSBLAS. 
